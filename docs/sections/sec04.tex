%----------------------------------------------------------------------------------------
%	PROGETTAZIONE
%----------------------------------------------------------------------------------------

\section{Progettazione}
Il processo di sviluppo adottato ha visto due fasi di progettazione ben distinte: il design architetturale e il design di dettaglio.\\ 

\subsection*{Design Architetturale}

Dapprima il gruppo si è concentrato sul design architetturale, ovvero la definizione dell'architettura generale del sistema, delle sue componenti principali e delle relazioni tra di esse.\\
Ciò ha permesso di individuare fin da subito gli elementi più importanti del sistema e di definire le interfacce tra di essi, nonchè di scegliere le tecnologie più appropriate
per supportare il core del sistema.\\

Considerazioni che hanno guidato la scelta dell'architettura:


Considerazioni che hanno guidato la scelta della tecnologia:
\begin{itemize}
    \item Il sistema è distribuito: vari componenti software, detti nodi, collocati su macchine diverse comunicano tra di loro attraverso una rete e realizzano un comportamento.
    \item Il sistema deve essere scalabile: deve essere semplice aggiungere e rimuovere nodi dal sistema.
    \item Il sistema deve essere robusto: deve essere in grado di tollerare guasti e malfunzionamenti di alcuni nodi.
\end{itemize}

\subsection*{Design di Dettaglio}


===============================================================================================================================

Devono essere esposte le scelte progettuali operate nelle varie fasi di sviluppo dell'elaborato.\\

In questa sezione devono essere documentati gli schemi di progetto relativamente all'architettura complessiva del sistema e alle sue componenti di rilievo che possano meritare un'analisi di dettaglio. Per le componenti software si può ricorrere ad esempio a diagrammi delle classi, di sequenza, stato, attività. Per le componenti hardware è possibile includere opportuni schemi in grado di descrivere l'architettura fisica adottata.\\

Vincoli circa la lunghezza della sezione (escluse didascalie, tabelle, testo nelle immagini, schemi):

\vspace{1cm}
\begin{tabular}{l|rr}
 & Numero minimo di battute & Numero massimo di battute \\
 \hline
 1 componente & 9000 & 18000 \\
 2 componenti & 12000 & 21000 \\
 3 componenti & 15000 & 24000 \\
 \hline
\end{tabular}


\newpage