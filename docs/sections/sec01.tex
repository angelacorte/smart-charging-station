
%----------------------------------------------------------------------------------------
%	INTRODUZIONE
%----------------------------------------------------------------------------------------

\section{Introduzione}

\subsection{Obiettivo del progetto}
L'obiettivo primario del progetto \textit{"Smart Charging Stations"} è rivoluzionare l'esperienza della
mobilità elettrica nelle smart cities attraverso l'implementazione di un sistema di gestione avanzato
per le stazioni di ricarica dei veicoli elettrici. Questa soluzione tecnologica mira a semplificare e
ottimizzare l'utilizzo delle stazioni di ricarica, fornendo agli utenti un accesso facile, rapido ed
efficiente alle risorse di ricarica, contribuendo così alla diffusione su larga scala dei veicoli elettrici
e alla promozione di una mobilità sostenibile.

\subsection{Caratteristiche salienti}
Il progetto \textit{"Smart Charging Stations"} si distingue per diverse caratteristiche innovative
che trasformeranno radicalmente il modo in cui interagiamo con le stazioni di ricarica dei veicoli elettrici.

In primo luogo, l'applicazione sviluppata permetterà agli utenti di individuare facilmente le
stazioni di ricarica disponibili all'interno della smart city attraverso una mappa interattiva.
Questa funzionalità si basa sull'integrazione di tecnologie di geolocalizzazione avanzate,
garantendo un'esperienza utente intuitiva e informata.

Un altro elemento chiave è la visualizzazione dello stato delle stazioni di ricarica.
Gli utenti potranno vedere in tempo reale se una stazione è occupata, prenotata o libera, il livello di
carica della macchina in ricarica. Questa trasparenza nell'informazione
consente agli utenti di pianificare efficacemente le loro attività di ricarica e di prendere decisioni
informate.

La possibilità di prenotare una stazione di ricarica attraverso l'applicazione è una caratteristica
che semplifica notevolmente la vita degli utenti. Ciò elimina la frustrazione di arrivare a una stazione
solo per trovarla occupata e garantisce un'esperienza senza intoppi. La prenotazione sarà possibile grazie
all'integrazione di tecnologie QR code e un sistema di gestione centralizzato.

\subsection{Contributo tecnologico-scientifico apportato}
Il nostro gruppo ha apportato un contributo sostanziale alla realizzazione del progetto
\textit{"Smart Charging Stations"}. Abbiamo progettato e sviluppato l'interfaccia utente
intuitiva dell'applicazione, focalizzandoci sulla user experience e sull'accessibilità.
Abbiamo integrato tecnologie di geolocalizzazione avanzate per consentire agli utenti di
individuare facilmente le stazioni di ricarica nelle vicinanze.

Inoltre, abbiamo implementato un sistema di gestione centralizzato che coordina
le prenotazioni delle stazioni di ricarica. Abbiamo sviluppato la funzionalità di sblocco delle stazioni
tramite QR code, garantendo un accesso sicuro e veloce alle risorse di ricarica.

Il nostro contributo scientifico ha riguardato l'ottimizzazione dei flussi di dati all'interno
dell'applicazione e
l'implementazione di protocolli di sicurezza per garantire la protezione dei dati sensibili
degli utenti.

In conclusione, il progetto \textit{"Smart Charging Stations"} rappresenta un passo avanti
significativo verso una mobilità elettrica più efficiente e accessibile nelle smart cities.
Grazie alle tecnologie integrate, all'interfaccia intuitiva e alle funzionalità avanzate,
ci proponiamo di favorire l'adozione su larga scala dei veicoli elettrici, contribuendo
così a un futuro più sostenibile e intelligente.


=====================================================
Esporre l’obiettivo del progetto dandone una visione complessiva.\\

Devono essere illustrate le caratteristiche salienti del progetto; deve essere chiara la distinzione tra le tecnologie usate/assemblate durante lo svolgimento dell'elaborato e il contributo tecnologico/scientifico effettivamente apportato dal gruppo.\\


Vincoli circa la lunghezza della sezione (escluse didascalie, tabelle, testo nelle immagini, schemi):

\vspace{1cm}
\begin{tabular}{l|rr}
                 & Numero minimo di battute & Numero massimo di battute \\
    \hline
    1 componente & 2000                     & 3000                      \\
    2 componenti & 2500                     & 4500                      \\
    3 componenti & 3000                     & 6000                      \\
    \hline
\end{tabular}


\newpage
