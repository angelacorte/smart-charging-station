%----------------------------------------------------------------------------------------
%	ANALISI DEI REQUISITI
%----------------------------------------------------------------------------------------

\section{Analisi dei requisiti}

\subsection*{Requisiti utente}

\begin{tabular}{l|rr}
    Codice requisito & Testo del requisito \\
    \hline
    1 & Visualizzazione dello stato di una singola stazione di ricarica \\
    1.1 & Stazione occupata o libera \\
    1.2 & Lo stato di completamento della ricarica \\
    1.3 & Stima della tempistica di completamento della ricarica \\
    1.4 & Real-time awareness della user app circa lo stato delle colonnine di ricarica \\
    2 & Visualizzazione di una mappa con le stazioni all'interno del territorio limitrofo \\
    2.1 & Visualizzazione dello stato (semplificato) di tutte le stazioni \\
    2.2 & Possibilità di selezionare una stazione nella mappa \\
    3 & Gestione delle stazioni di ricarica \\
    3.1 & Prenotazione di una stazione di ricarica \\
    3.2 & Aggiunta di stazione d'interesse \\
    3.3 & Ricezione di una notifica nel caso in cui si liberi una stazione d'interesse \\
    3.4 & Segnalazione di eventuali problemi inerenti ad una stazione di ricarica \\
    3.5 & Sblocco di una colonnina con QR code e fotocamera \\
    4 & Pagamento ricarica in app \\
    5 & Interfacciamento con servizi esterni tramite http
   \end{tabular}
   
\subsection*{Requisiti funzionali}
In questa sezione esporre brevemente i requisiti a cui il sistema proposto deve rispondere, concentrando l'attenzione sugli aspetti più rilevanti e facendo eventualmente uso di opportuni diagrammi di alto livello.\\

Vincoli circa la lunghezza della sezione (escluse didascalie, tabelle, testo nelle immagini, schemi):

\vspace{1cm}
\begin{tabular}{l|rr}
 & Numero minimo di battute & Numero massimo di battute \\
 \hline
 1 componente & 4000 & 6000 \\
 2 componenti & 6000 & 8000 \\
 3 componenti & 8000 & 10000 \\
 \hline
\end{tabular}


\newpage
